This page describes the basic concepts and setup that are needed for all types of Java modules. It is not needed for Python module development.\hypertarget{mod_dev_page_mod_dev_setup}{}\section{Basic Setup}\label{mod_dev_page_mod_dev_setup}
\hypertarget{mod_dev_page_mod_dev_setup_nb}{}\subsection{Net\+Beans and Java}\label{mod_dev_page_mod_dev_setup_nb}
Autopsy is built on top of the Net\+Beans Rich Client Platform, which makes it easy to make plug-\/in infrastructures. To do any development, you really need to download Net\+Beans first. You can in theory develop modules by command line only, but this document assumes that you are using the I\+DE. Download and install the latest version of the I\+DE from \href{http://www.netbeans.org}{\tt http\+://www.\+netbeans.\+org}.

Autopsy currently requires Java 1.\+8. Ensure that it is installed.\hypertarget{mod_dev_page_mod_dev_setup_platform}{}\subsection{Obtain the Autopsy Platform}\label{mod_dev_page_mod_dev_setup_platform}
Before we can make a module, we must configure Net\+Beans to know about Autopsy as a platform. This will allow you to access all of the classes and services that Autopsy provides. There are two ways of configuring the Net\+Beans I\+DE to know about Autopsy\+:


\begin{DoxyItemize}
\item Download an official release of Autopsy and build against it.
\item Download Autopsy source code, build it, and make a platform to build against.
\end{DoxyItemize}\hypertarget{mod_dev_page_mod_dev_setup_platform_rel}{}\subsubsection{Using a Released Version}\label{mod_dev_page_mod_dev_setup_platform_rel}
The easiest method for obtaining the platform is to install Autopsy on your computer. It will have everything that you need. If you installed it in \char`\"{}\+C\+:\textbackslash{}\+Program Files\textbackslash{}\+Autopsy\char`\"{}, then the platform is in \char`\"{}\+C\+:\textbackslash{}\+Program Files\textbackslash{}\+Autopsy\textbackslash{}platform\char`\"{}. You can now also download just the Z\+IP file of the Autopsy release instead of the M\+SI installer. This maybe more convenient for development situations.\hypertarget{mod_dev_page_mod_dev_setup_platform_src}{}\subsubsection{Building a Platform from Code}\label{mod_dev_page_mod_dev_setup_platform_src}
If you want to build against the bleeding edge code and updates that have occurred since the last release, then you must download the latest source code and build it. This involves getting a full development environment setup. Refer to the wiki page at \href{http://wiki.sleuthkit.org/index.php?title=Autopsy_Developer%27s_Guide}{\tt http\+://wiki.\+sleuthkit.\+org/index.\+php?title=\+Autopsy\+\_\+\+Developer\%27s\+\_\+\+Guide} for details on getting the source code and a development environment setup.

To use the latest Autopsy source code as your development environment, first follow B\+U\+I\+L\+D\+I\+N\+G.\+T\+XT in the root source repository to properly build and setup Autopsy in Net\+Beans.

Once Autopsy has been successfully built, right click on the Autopsy project in Net\+Beans and select Package as $>$ Z\+IP Distribution. Once the Z\+IP file is created, extract its contents to a directory. This directory is the platform that you will build against. Note that you will building the module against this built platform. If you need to make changes to Autopsy infrastructure for your module, then you will need to then make a new Z\+IP file and configure your module to use it each time.\hypertarget{mod_dev_page_mod_dev_module}{}\section{Creating a Basic Net\+Beans Module}\label{mod_dev_page_mod_dev_module}
The Autopsy modules are encapsulated inside of Net\+Beans modules. A Net\+Beans module will be packaged as a single \char`\"{}.\+nbm\char`\"{} file. A single Net\+Beans module can contain many Autopsy modules. The Net\+Beans module is what the user will install and provides things like auto-\/update.\hypertarget{mod_dev_page_mod_dev_mod_nb}{}\subsection{Creating a Net\+Beans Module}\label{mod_dev_page_mod_dev_mod_nb}
If this is your first module, then you will need to make a Net\+Beans module. If you have already made an Autopsy module and are now working on a second one, you can consider adding it to your previous Net\+Beans module.

To make a Net\+Beans module\+:
\begin{DoxyItemize}
\item Open the Net\+Beans I\+DE and go to File -\/$>$ New Project.
\item From the list of categories, choose \char`\"{}\+Net\+Beans Modules\char`\"{} and then \char`\"{}\+Module\char`\"{} from the list of \char`\"{}\+Projects\char`\"{}. Click Next.
\item In the next panel of the wizard, give the module a name and directory. Select Standalone Module (the default is typically \char`\"{}\+Add to Suite\char`\"{}) so that you build the module as an external module against Autopsy. You will need to tell Net\+Beans about the Autopsy platform, so choose the \char`\"{}\+Manage\char`\"{} button. Choose the \char`\"{}\+Add Platform\char`\"{} button and browse to the location of the platform discussed in the previous sections (as a reminder this will either be the location that you installed Autopsy into or where you opened up the Z\+IP file you created from source). Click Next.
\item Finally, enter the code base name. We use the same naming convention as used for naming packages (\href{http://docs.oracle.com/javase/tutorial/java/package/namingpkgs.html}{\tt http\+://docs.\+oracle.\+com/javase/tutorial/java/package/namingpkgs.\+html}). Press Finish.
\end{DoxyItemize}\hypertarget{mod_dev_page_mod_dev_mod_nb_config}{}\subsubsection{Configuring the Net\+Beans Module}\label{mod_dev_page_mod_dev_mod_nb_config}
After the module is created, you will need to do some further configuration.
\begin{DoxyItemize}
\item Right click on the newly created module and choose \char`\"{}\+Properties\char`\"{}.
\item You will need to configure the module to be dependent on modules from within the Autopsy platform. Go to the \char`\"{}\+Libraries\char`\"{} area and choose \char`\"{}\+Add\char`\"{} in the \char`\"{}\+Module Dependencies\char`\"{} section. Choose the\+: -- \char`\"{}\+Autopsy-\/core\char`\"{} library to get access to the Autopsy services. -- \char`\"{}\+Net\+Beans Lookup\char`\"{} library so that your module can be discovered by Autopsy.
\item If you later determine that you need to pull in external J\+AR files, then you will use the \char`\"{}\+Wrapped Jar\char`\"{} section to add them in.
\item Note, you will also need to come back to this section if you update the platform. You may need to add a new dependency for the version of the Autopsy-\/core that comes with the updated platform.
\item Autopsy requires that all modules restart Autopsy after they are installed. Configure your module this way under Build -\/$>$ Packaging. Check the box that says Needs Restart on Install.
\end{DoxyItemize}

You now have a Net\+Beans module that is using Autopsy as its build platform. That means you will have access to all of the services and utilities that Autopsy provides (such as services\+\_\+page).\hypertarget{mod_dev_page_mod_dev_mod_config_other}{}\subsubsection{Optional Settings}\label{mod_dev_page_mod_dev_mod_config_other}
There are several optional things in the Properties section. You can add a description and specify the version. You can do all of this later though and it does not need to be done before you start development.

A link about the Net\+Beans versioning scheme can be found here \href{http://wiki.netbeans.org/VersioningPolicy}{\tt http\+://wiki.\+netbeans.\+org/\+Versioning\+Policy}. Autopsy follows this scheme and a link to the details can be found at \href{http://wiki.sleuthkit.org/index.php?title=Autopsy_3_Module_Versions}{\tt http\+://wiki.\+sleuthkit.\+org/index.\+php?title=\+Autopsy\+\_\+3\+\_\+\+Module\+\_\+\+Versions}.\hypertarget{mod_dev_page_mod_dev_mod_other}{}\subsection{Other Links}\label{mod_dev_page_mod_dev_mod_other}
For general Net\+Beans module information, refer to \href{http://bits.netbeans.org/dev/javadoc/org-openide-modules/org/openide/modules/doc-files/api.html}{\tt this guide from Net\+Beans.\+org}.\hypertarget{mod_dev_page_mod_dev_aut}{}\section{Creating Autopsy Modules}\label{mod_dev_page_mod_dev_aut}
You can now add Autopsy modules into the Net\+Beans container module. There are other pages that focus on that and are listed on the main page. The rest of this document contains info that you will eventually want to come back to though. As you will read in the later sections about the different module types, each Autopsy Module is a java class that extends an interface (the interface depends on the type of module).\hypertarget{mod_dev_page_mod_dev_aut_run1}{}\subsection{Running Your Module During Development}\label{mod_dev_page_mod_dev_aut_run1}
When you are developing your Autopsy module, you can simply choose \char`\"{}\+Run\char`\"{} on the module and it will launch the Autopsy platform with the module enabled in it. This is also how you can debug the module.\hypertarget{mod_dev_page_mod_dev_aut_deploy}{}\subsection{Deploying Your Module}\label{mod_dev_page_mod_dev_aut_deploy}
When you are ready to share your module, create an N\+BM file by right clicking on the module and selecting \char`\"{}\+Create N\+B\+M\char`\"{}.\hypertarget{mod_dev_page_mod_dev_aut_install}{}\subsection{Installing Your Module}\label{mod_dev_page_mod_dev_aut_install}
To install the module on a non-\/development environment, launch Autopsy and choose Plugins under the Tools menu. Open the Downloaded tab and click Add Plugins. Navigate to the N\+BM file and open it. Next, click Install and follow the wizard. 