\hypertarget{regression_test_page_regression_test_overview}{}\section{Overview}\label{regression_test_page_regression_test_overview}
Autopsy uses Netbeans and Jelly testing framework for regression testing. Testing is driven by invoking UI actions via Jelly framework.

Currently, Autopsy regression testing automates the following\+:


\begin{DoxyItemize}
\item creating a case
\item adding an image
\item configuring ingest
\item running ingest
\item generating an H\+T\+ML report
\end{DoxyItemize}

The tests can be invoked using ant regression-\/test

There is a python script in Testing/script/regression.\+py that wraps around \char`\"{}ant regression-\/test\char`\"{} and runs a test for every disk image it finds.

regression.\+py also does regression test result validation by comparing the test result with the golden standards for the image and comparing the H\+T\+ML report with the gold standard H\+T\+ML report.

It is assumed that the steps detailed in the Building Autopsy from Source document have been completed, and that appropriate versions of the J\+DK, L\+I\+B\+E\+WF etc, are present on the system. Building Autopsy from Source can be accessed at\+: \href{https://github.com/sleuthkit/autopsy/blob/master/BUILDING.txt}{\tt https\+://github.\+com/sleuthkit/autopsy/blob/master/\+B\+U\+I\+L\+D\+I\+N\+G.\+txt}\hypertarget{regression_test_page_regression_test_setup}{}\section{Setting up regression testing}\label{regression_test_page_regression_test_setup}
1) Install Cygwin

Download and run the Cygwin setup, available at \begin{DoxyVerb}        http://www.cygwin.com/setup.exe
\end{DoxyVerb}


From the list of packages to install, select both Database and Python.

2) Setting up regression.\+py

If you are planning on running the script regularly (\char`\"{}./regression.\+py\char`\"{}), the input directory will be \begin{DoxyVerb}    autopsy/Testing/script/input
\end{DoxyVerb}


However, if you are planning on using the configuration file, you can change the input directory using the $<$indir$>$ tag.

Ensure the following files are in the input directory, whether in the default location or set by the configuration file \begin{DoxyVerb}    notablehashes.txt-md5.idx
    nsrl.txt-md5.idx
    notablekeywords.xml
\end{DoxyVerb}


Place any images you would like to test in the input directory, and/or use the configuration file to point to image files directly using the $<$image$>$ tag. The input directory will be scanned for images even when the configuration file points to specific ones. Use the -\/i or --ignore command to prevent this.\hypertarget{regression_test_page_regression_test_running}{}\section{Running regression testing}\label{regression_test_page_regression_test_running}
3) Running regression.\+py

From the Cygwin terminal, navigate to \begin{DoxyVerb}autopsy/Testing/script
\end{DoxyVerb}


To run regression.\+py using the default settings, type \begin{DoxyVerb}./regression.py
\end{DoxyVerb}


By default this will


\begin{DoxyItemize}
\item Search for notablehashes.\+txt-\/md5.\+idx, nsrl.\+txt-\/md5.\+idx, and notablekeywords.\+xml in the ./input directory
\item Search for any image files in the ./input directory and test them
\item Compare the generated database and report to the gold standards
\item Save Autopsy\+Errors.\+txt, which is a log of all Warnings and Exceptions thrown by Autopsy, in the output directory
\item Save C\+S\+V.\+txt, which contains all important test detailes for each image delimited by a \char`\"{}$\vert$\char`\"{}, in the output directory
\item Save Autopsy\+Test\+Case.\+html, which is a user-\/friendly information, warning, and command-\/line log of all the tests ran, in the output directory
\item Delete the Solr index in the output directory to save drive space
\end{DoxyItemize}

Other commands can be issued by adding the following arguments to the script \begin{DoxyVerb}-r, --rebuild
\end{DoxyVerb}
 Rebuild the gold standards from the test results for each image. -\/i, --ignore Ignores the ./input directory when searching for files. Only use in combination with a configuration file. -\/u, --unallocated Ignores unallocated space when ingesting. Faster, but yields less accurate results. -\/k, --keep Keeps each image\textquotesingle{}s Solr index instead of deleting it. -\/v, --verbose Prints all Warnings and Exceptions after each ingest. -\/e, --exception When followed by a string, will print out all exceptions that occured that contain the string. Case sensitive. -\/l, --list Runs from a configuration file, which is given as a path to the file after the argument. -\/c, --continuous Runs the arguments from the configuration file in a loop until interrupted, must be used with -\/l.\hypertarget{regression_test_page_regression_test_config}{}\section{Running regression tests with a config file}\label{regression_test_page_regression_test_config}
4) Running from a configuration file

An X\+ML configuration file can be pointed to by using the -\/l or --list command as shown below \begin{DoxyVerb}./regression.py -l X:\path\to\file.xml
\end{DoxyVerb}


The configuration file has three possible tags, all of which are optional.

$<$indir value=\char`\"{}\+X\+:\textbackslash{}path\textbackslash{}to\textbackslash{}input\textbackslash{}directory\char`\"{}$>$ Changes the default input directory to the one provided.

$<$global\+\_\+csv value=\char`\"{}\+X\+:\textbackslash{}path\textbackslash{}to\textbackslash{}csv\textbackslash{}file.\+txt\char`\"{}$>$ Points to an optional C\+SV file to be added to, anywhere in the user\textquotesingle{}s filesystem.

$<$image value=\char`\"{}\+X\+:\textbackslash{}path\textbackslash{}to\textbackslash{}image.\+img\char`\"{}$>$ Points to a specific image file to be tested. This tag can be used any number of times in the configuration file.

Note\+: When running with a configuration file, all the default features of regression.\+py will still be enabled.

Please see the example configuration file \begin{DoxyVerb}autopsy/Testing/script/config.xml
\end{DoxyVerb}


For more detail

/section optional\+\_\+standards\+\_\+update O\+P\+T\+I\+O\+N\+AL\+: Update the standards databases

From the Cygwin terminal, navigate to \begin{DoxyVerb}autopsy/Testing/script
\end{DoxyVerb}


run\+: \begin{DoxyVerb}./regression.py -r
\end{DoxyVerb}


The script will automatically delete pre-\/existing standards.\+db files and generate the updated ones in the proper locations (/script/gold/\{name of image\}).

Running in -\/r will also generate a golden report file built from the image. Normal runs of regression.\+py compare their generated report against the golden one, and report any differences in the file, ignoring the timestamp.\hypertarget{regression_test_page_developers_note_regression_test}{}\section{Developers Notes\+: Jemmy and Regression\+Test.\+java}\label{regression_test_page_developers_note_regression_test}
For additional details regarding setting up and using Jemmy, please see \begin{DoxyVerb}http://platform.netbeans.org/tutorials/nbm-test.html
http://wiki.netbeans.org/Writing_JellyTools_Tests_Guide
\end{DoxyVerb}


The Jemmy UI framework includes elements such as buttons, frames, dialog boxes and wizards. In order to manipulate these elements programmatically, the associated Container\+Operators must be used. Regression\+Test.\+java makes use of the following major operators\+: \begin{DoxyVerb}JButtonOperator
JDialogOperator
nbDialogOperator
JTableOperator
JFileChooserOperator
WizardOperator
\end{DoxyVerb}


Wizard\+Operators are for elements that implement the Wizard interface. Wizards specifically have back and next buttons. A Wizard\+Operator can be created by \begin{DoxyVerb}WizardOperator wo = new WizardOperator(String title);
\end{DoxyVerb}


Where title is the display title of the wizard you wish to manipulate.

In order to use any Jemmy UI element, it must first be found. There are a number of ways to do this, but the most common involves searching by the display name of the element in question. Finding elements is a function of that elements Container\+Operator. For example, to find a J\+Dialog whose display name is the string \char`\"{}\+Hash Database Configuration\char`\"{}, the following code might be used\+: \begin{DoxyVerb}JDialog hashMainDialog =  JDialogOperator.waitJDialog("Hash Database Configuration", false, false);
\end{DoxyVerb}


The two booleans are for searching the exact string including substrings, and for searching case sensitively.

Note that the method used is called \textquotesingle{}.wait\+J\+Dialog\textquotesingle{}, and not \textquotesingle{}.find\+J\+Dialog\textquotesingle{}. This is an important distinction regarding thoroughness of the find, but the functionality of the same. Refer to the link on Jemmy above for greater detail.

Once you an element has been located, it can be operated upon by creating a new Container\+Operator, with the element as the only argument\+: \begin{DoxyVerb}JDialogOperator hashMainDialogOperator = new JDialogOperator(hashMainDialog);
\end{DoxyVerb}


Selecting the main window\+:

In order to select the main window, in this case, the general Autospy frame, the Main\+Window\+Operator must be used. A Main\+Window\+Operator takes no arguments and is created as follows\+: \begin{DoxyVerb}MainWindowOperator mwo = MainWindowOperator.getDefault();
\end{DoxyVerb}


For further reference regarding Container\+Operators, please see \begin{DoxyVerb}http://www.jarvana.com/jarvana/view/org/netbeans/jemmy/2.2.7.5/jemmy-2.2.7.5-javadoc.jar!/org/netbeans/jemmy/operators/ContainerOperator.html
\end{DoxyVerb}


When an element has been selected, the individual components may be manipulated with Container\+Operators. To select a button, use the code below, where cont is one of the Container\+Operators from above, text is the text displayed on the button, and index is the button\textquotesingle{}s order if there are multiple with the same name (i.\+e. if there are three buttons labeled �preview�, the first\textquotesingle{}s index is 0, then 1, then 2).

Jbutton\+Operator jbo = new Jbutton\+Operator(\+Container\+Operator cont, String text, int index);

There are many others elements and operators, such as Jcheck\+Box\+Operator, Jfile\+Chooser\+Operator, Jtext\+Field\+Operator, etc. See \href{http://www.jarvana.com/jarvana/view/org/netbeans/jemmy/2.2.7.5/jemmy-2.2.7.5-javadoc.jar!/org/netbeans/jemmy/operators/JComponentOperator.html}{\tt http\+://www.\+jarvana.\+com/jarvana/view/org/netbeans/jemmy/2.\+2.\+7.\+5/jemmy-\/2.\+2.\+7.\+5-\/javadoc.\+jar!/org/netbeans/jemmy/operators/\+J\+Component\+Operator.\+html} for more. Please see their individual Java\+Docs for action commands that push buttons, write in forms, etc.

If an element cannot be grabbed using a Container\+Operator, a temporary workaround is to invoke the element action\+: \begin{DoxyVerb}new Action(String menuPath, String popupPath).perform();
\end{DoxyVerb}


where menu\+Path is the path through the File menu to said action and popup is the path through the popup menu (which is null since it is unsupported).

For more on Actions, see \begin{DoxyVerb}http://bits.netbeans.org/dev/javadoc/org-netbeans-modules-jellytools-platform/org/netbeans/jellytools/actions/Action.html\end{DoxyVerb}
 