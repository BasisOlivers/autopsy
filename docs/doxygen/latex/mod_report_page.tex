\hypertarget{mod_report_page_report_summary}{}\section{Overview}\label{mod_report_page_report_summary}
Report modules allow Autopsy users to create different report types. Autopsy comes with modules to generate H\+T\+ML and Excel artifact reports, a tab delimited File report, a Keyhole Markup Language (K\+ML) report for Google Earth data, and a body file for timeline creation. You can make additional modules to create custom output formats.

There are three types of reporting modules that differ in how the data is organized.
\begin{DoxyItemize}
\item Table report modules organize the data into tables. If your output is in table format, this type of module will be easier to make because Autopsy does a lot of the organizing work for you.
\item File report modules are also table-\/based, but they specifically deal with reporting on the Files in the case, not artifacts.
\item General report modules are free form and you are allowed to organize the output however you want.
\end{DoxyItemize}

Table report modules require their subclasses to override methods to start and end tables, and add rows to those tables. These methods are provided data, generated from a default configuration panel, for the module to report on. Because of this, when creating a table report module one only needs to focus on how to display the data, not how to find it.

File report modules are similar to table report modules, but only require their sub-\/classes to start and end a single table, and add rows to that table. The methods are given an Abstract\+File and a list of File\+Report\+Data\+Types, which specify what information about the file should be added to the report. The data can be extracted from the file by calling the File\+Report\+Data\+Types get\+Value method with the file as it\textquotesingle{}s argument.

On the other hand, general report modules have a single method to generate the report. This method gives the module freedom to find and process any data it so chooses. General modules also have the ability to provide a configuration panel, allowing the user to choose from various displayed settings. The report module may then use the user\textquotesingle{}s selection to generate a more specific report.

General modules are also given the responsibility of updating their report\textquotesingle{}s progress bar and processing label in the UI. A progress panel is given to every general report module. It contains basic A\+PI to start, stop, and add to the progress bar, as well as update the processing label. The module is also expected to check the progress bar\textquotesingle{}s status occasionally to see if the user has manually canceled the report.\hypertarget{mod_report_page_report_create_module}{}\section{Creating a Report Module}\label{mod_report_page_report_create_module}
To create a report module, start off by creating a new Java or Python (Jython) class and implementing (Java) or inheriting (Jython) the appropriate interface\+:
\begin{DoxyItemize}
\item org.\+sleuthkit.\+autopsy.\+report.\+Table\+Report\+Module
\item org.\+sleuthkit.\+autopsy.\+report.\+File\+Report\+Module
\item org.\+sleuthkit.\+autopsy.\+report.\+General\+Report\+Module
\end{DoxyItemize}

All three of these interfaces extend the org.\+sleuthkit.\+autopsy.\+report.\+Report\+Module interface that defines the following methods\+:
\begin{DoxyItemize}
\item org.\+sleuthkit.\+autopsy.\+report.\+Report\+Module.\+get\+Name()
\item org.\+sleuthkit.\+autopsy.\+report.\+Report\+Module.\+get\+Description()
\item org.\+sleuthkit.\+autopsy.\+report.\+Report\+Module.\+get\+Relative\+File\+Path()
\end{DoxyItemize}

These methods will be called by Autopsy when it is presenting the reporting UI to a user.\hypertarget{mod_report_page_report_create_module_table}{}\subsection{Creating A Table Report Module}\label{mod_report_page_report_create_module_table}
If you implement Table\+Report\+Module, you should override the methods\+:
\begin{DoxyItemize}
\item org.\+sleuthkit.\+autopsy.\+report.\+Table\+Report\+Module.\+start\+Report(\+String path)
\item org.\+sleuthkit.\+autopsy.\+report.\+Table\+Report\+Module.\+end\+Report()
\item org.\+sleuthkit.\+autopsy.\+report.\+Table\+Report\+Module.\+start\+Data\+Type(\+String title)
\item org.\+sleuthkit.\+autopsy.\+report.\+Table\+Report\+Module.\+end\+Data\+Type()
\item org.\+sleuthkit.\+autopsy.\+report.\+Table\+Report\+Module.\+start\+Set(\+String set\+Name)
\item org.\+sleuthkit.\+autopsy.\+report.\+Table\+Report\+Module.\+end\+Set()
\item org.\+sleuthkit.\+autopsy.\+report.\+Table\+Report\+Module.\+add\+Set\+Index(\+List$<$\+String$>$ sets)
\item org.\+sleuthkit.\+autopsy.\+report.\+Table\+Report\+Module.\+add\+Set\+Element(\+String element\+Name)
\item org.\+sleuthkit.\+autopsy.\+report.\+Table\+Report\+Module.\+start\+Table(\+List$<$\+String$>$ titles)
\item org.\+sleuthkit.\+autopsy.\+report.\+Table\+Report\+Module.\+end\+Table()
\item org.\+sleuthkit.\+autopsy.\+report.\+Table\+Report\+Module.\+add\+Row(\+List$<$\+String$>$ row)
\item org.\+sleuthkit.\+autopsy.\+report.\+Table\+Report\+Module.\+date\+To\+String(long date)
\end{DoxyItemize}

When generating table module reports, Autopsy will iterate through a list of user selected data, and call methods such as add\+Row(\+List$<$\+String$>$ row) for every \char`\"{}row\char`\"{} of data it finds, or start\+Table(\+List$<$\+String$>$ titles) for every new category it finds. Developers are guaranteed that every start of a data type, set, or table will be followed by an appropriate end. The focus for a table report module should be to take the given information and display it in a user friendly format. See org.\+sleuthkit.\+autopsy.\+report.\+Report\+Excel for an example.\hypertarget{mod_report_page_report_create_module_file}{}\subsection{Creating a File Report Module}\label{mod_report_page_report_create_module_file}
If you implement File\+Report\+Module, the overriden methods will be\+:
\begin{DoxyItemize}
\item org.\+sleuthkit.\+autopsy.\+report.\+File\+Report\+Module.\+start\+Report(\+String path)
\item org.\+sleuthkit.\+autopsy.\+report.\+File\+Report\+Module.\+end\+Report()
\item org.\+sleuthkit.\+autopsy.\+report.\+File\+Report\+Module.\+start\+Table(\+List$<$\+File\+Report\+Data\+Types$>$ headers)
\item org.\+sleuthkit.\+autopsy.\+report.\+File\+Report\+Module.\+end\+Table()
\item org.\+sleuthkit.\+autopsy.\+report.\+File\+Report\+Module.\+add\+Row(\+Abstract\+File to\+Add, List$<$\+File\+Report\+Data\+Types$>$ columns)
\end{DoxyItemize}

As when generating table module reports, Autopsy will iterate through a list of user selected data (which are represented by File\+Report\+Data\+Types), and call add\+Row(\+Abstract\+File to\+Add, List$<$\+File\+Report\+Data\+Types$>$ columns) for every abstract file in the case. Developers are guaranteed that the order of method calls will be start\+Report(), start\+Table(\+List$<$\+File\+Report\+Data\+Types$>$ headers), add\+Row(\+Abstract\+File to\+Add, List$<$\+File\+Report\+Data\+Types$>$ columns), Abstract\+File to\+Add, List$<$\+File\+Report\+Data\+Types$>$ columns),..., end\+Table(), end\+Report().\hypertarget{mod_report_page_report_create_module_general}{}\subsection{Creating a General Report Module}\label{mod_report_page_report_create_module_general}
If you implement General\+Report\+Module, the overriden methods will be\+:
\begin{DoxyItemize}
\item org.\+sleuthkit.\+autopsy.\+report.\+General\+Report\+Module.\+generate\+Report(\+String report\+Path, Report\+Progress\+Panel progress\+Panel)
\item org.\+sleuthkit.\+autopsy.\+report.\+General\+Report\+Module.\+get\+Configuration\+Panel()
\end{DoxyItemize}

For general report modules, Autopsy will simply call the generate\+Report(\+String report\+Path, Report\+Progress\+Panel progress\+Panel) method and leave it up to the module to aquire and report data in its desired format. The only requirements are that the module saves to the given report path and updates the org.\+sleuthkit.\+autopsy.\+report.\+Report\+Progress\+Panel as the report progresses.

When updating the progress panel, it is recommended to update it as infrequently as possible, while still keeping the user informed. If your report processes 100,000 files and you chose to update the UI each time a file is reviewed, the UI would freeze when trying to process all your requests. This would cause problems to not only your reporting module, but to other modules running in parallel. A safer approach would be to update the UI every 1,000 files, or when a certain \char`\"{}category\char`\"{} of the files being processed has changed. For example, the H\+T\+ML report module increments the progress bar and changes the processing label every time a new Blackboard Artifact Type is being processed.

Autopsy will also display the panel returned by get\+Configuration\+Panel() in the generate report wizard. This panel can be used to allow the user custom controls over the report. To make this panel, use Net\+Beans to make a new J\+Panel class and use its layout interface to put the UI widgets in the places that you want them. Then, your get\+Configuration\+Panel() method should create an instance of that class and return it.

Typically a general report module should interact with both the Blackboard A\+PI in the org.\+sleuthkit.\+datamodel.\+Sleuthkit\+Case class, in addition to an A\+PI (possibly external/thirdparty) to convert Blackboard Artifacts to the desired reporting format.\hypertarget{mod_report_page_report_create_module_showing}{}\subsection{Showing Results}\label{mod_report_page_report_create_module_showing}
You should call Case.\+add\+Report() with the path to your report so that it is shown in the Autopsy tree. You can specify a specific file or folder and the user can then view it later.\hypertarget{mod_report_page_report_create_module_layer}{}\subsection{Installing your Report Module}\label{mod_report_page_report_create_module_layer}
Report modules developed using Java must be registered in a layer.\+xml file. This file allows Autopsy to find the report module.

An example entry in a layer.\+xml is shown below\+: 
\begin{DoxyCode}
<folder name=\textcolor{stringliteral}{"Services"}>
    <file name=\textcolor{stringliteral}{"org-sleuthkit-autopsy-report-ReportHTML.instance"}>
        <attr name=\textcolor{stringliteral}{"instanceOf"} stringvalue=\textcolor{stringliteral}{"org.sleuthkit.autopsy.report.TableReportModule"}/>
        <attr name=\textcolor{stringliteral}{"instanceCreate"} methodvalue=\textcolor{stringliteral}{"org.sleuthkit.autopsy.report.ReportHTML.getDefault"}/>
        <attr name=\textcolor{stringliteral}{"position"} intvalue=\textcolor{stringliteral}{"910"}/>
    </file>
</folder>
\end{DoxyCode}


In the above example, \char`\"{}org-\/sleuthkit-\/autopsy-\/report-\/\+Report\+H\+T\+M\+L\char`\"{} should be replaced with the package path to your report module.

It is also important to remember to include a get\+Default() method in your report module. As shown in the code above, the instance to each report module is accessed via it\textquotesingle{}s get\+Default() method. For example\+:


\begin{DoxyCode}
\textcolor{comment}{// Static instance of this report}
\textcolor{keyword}{private} \textcolor{keyword}{static} MyReport instance;

\textcolor{comment}{// Get the default instance of this report}
\textcolor{keyword}{public} \textcolor{keyword}{static} \textcolor{keyword}{synchronized} MyReport getDefault() \{
    \textcolor{keywordflow}{if} (instance == null) \{
        instance = \textcolor{keyword}{new} MyReport();
    \}
    \textcolor{keywordflow}{return} instance;
\}
\end{DoxyCode}


Report modules developed using Jython are installed in Autopsy by placing them in their own subdirectory of the python\+\_\+modules directory. A window into the python\+\_\+modules directory can be opened through the Autopsy\textquotesingle{}s Tools -\/$>$ Python Plugins menu item. Create a folder in this directory and create or place your Python scripts in this folder. 